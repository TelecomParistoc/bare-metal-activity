\documentclass[a4paper,10pt]{article} %type de document et paramètres

\usepackage{lmodern} %police de caractère
\usepackage[english,french]{babel} %package de langues
\usepackage[utf8]{inputenc} %package fondamental
\usepackage[T1]{fontenc} %package fondamental

\usepackage[top=3cm, bottom=3cm, left=3cm, right=3cm]{geometry} %permet de paramétrer les marges par défaut
\usepackage{changepage} %permet de modifier localement une mise en page (marges,...) : utilisé pour la page de garde
%\usepackage{multicol} %permet de mettre plusieurs colonnes (\begin{multicols}{2} \end{multicols} jusqu'à 10 colonnes)
\usepackage[pdftex, pdfauthor={Pierre Gimalac}, pdftitle={Activité bare-metal}, colorlinks=false, linkcolor=black]{hyperref} %permet de se déplacer dans le pdf depuis le sommaire en cliquant sur les titres, ainsi que de parametrer les meta données du PDF
\usepackage{url} %permet de mettre des URL actifs \url{}
\let\urlorig\url
\renewcommand{\url}[1]{\begin{otherlanguage}{english}\urlorig{#1}\end{otherlanguage}}

%\usepackage{mathtools} %maths (à developper, utile par exemple pour enlever les espaces dus aux boites $\sum_{\mathclap{1\le i\le j\le n}} X_{ij}$)
%\usepackage{amssymb} %maths
%\usepackage{amsthm} %maths
%\usepackage{amsmath} %maths
%\usepackage{mathrsfs} %maths (par exemple les lettres caligraphiées)
%\usepackage{stmaryrd} %maths (par exemple les ensembles d'entiers \rrbracket \llbracket)
%\usepackage{calrsfs} %maths (par exemple les notations des ensembles)
%\usepackage{yhmath} % permet de noter les arcs de cercle avec \wideparen{AOB}
%\usepackage{xlop} %permet d'afficher des opérations mathématiques
%\usepackage[squaren,Gray]{SIunits} %permet de noter des unités proprement
%\usepackage{esdiff} %permet d'écrire la dérivée avec la notation de Leibniz \diff{v}{t}

\usepackage{graphicx} %permet d'insérer des images proprement (ajoute des parametres)
\usepackage{wrapfig} %permet de mettre des images à coté d'un texte
%\usepackage{pdfpages} %permet d'insérer un pdf \includepdf[pages={1-2}]{truc.pdf}
\usepackage{enumitem} %permet de changer le label d'une liste \begin{itemize}[label=$\cdot$]
% \usepackage{ulem} %permet de souligner/barrer du texte
%\usepackage{soul} %permet de souligner/barrer du texte
% \usepackage{cancel} %permet de barrer du texte /cancel{text}


%\usepackage{tikz} %package trooop bien permet de dessiner tout et n'importe quoi ! \begin{tikzpicture}
%\usetikzlibrary{automata,positioning} % pour dessiner des automates
%\usepackage{circuitikz} %permet de dessiner des circuits logiques (entre autre) avec la syntaxe de tikz (\begin{circuitikz}) par exemple \node[american not port] pour le 'non'
%\usepackage{listings} %permet d'inserer du code dans le fichier (\lstset{language=Java} \begin{lstlisting} \end{lstlisting} )


%\newcommand{\R}{\mathbb{R}}
%\newcommand{\Rpe}{\mathbb{R}_{+}^{*}}
%\newcommand{\Rb}{\overline{\mathbb{R}}}
%\newcommand{\N}{\mathbb{N}}
%\newcommand{\Z}{\mathbb{Z}}
%\newcommand{\C}{\mathbb{C}}
%\newcommand{\Q}{\mathbb{Q}}
%\newcommand{\K}{\mathbb{K}}
%\newcommand{\E}{\mathbb{E}} % espérance
%\renewcommand{\P}{\mathbb{P}} % fonction de probabilité
%\newcommand{\F}{\mathbb{F}}
%\newcommand\abs[1]{\left|#1\right|}
%\newcommand{\tq}{~|~}
%\newcommand\fra[2]{\genfrac{}{}{0pt}{1}{#1}{#2}}
%\newcommand\equi[1]{\renewcommand{\arraystretch}{0.3}~\begin{matrix}\sim\\#1\end{matrix}~\renewcommand{\arraystretch}{1}}
%\newcommand{\dl}{développement limité }
%\newcommand{\dls}{développements limités }
%\newcommand{\ev}{espace vectoriel }
%\newcommand{\evs}{espaces vectoriels }
%\newcommand{\sev}{sous-espace vectoriel }
%\newcommand{\sevs}{sous-espaces vectoriels }
%\newcommand{\displayAmath}{\displaystyle}
%\newcommand{\lime}[4]{#1\underset{\mathclap{#2 \rightarrow #3}}{\longrightarrow} #4}
%\newcommand{\supp}{\mathrm{supp}~} % support
%\newcommand{\Ima}{\mathrm{Im}~} % image
%\newcommand{\Inv}{\mathrm{Inv}~} % nombre d'inversion d'une permutation
%\newcommand{\ord}{\mathrm{ord}~} % ordre
%\newcommand{\com}{\mathrm{com}~} % comatrice
%\newcommand{\oversim}[1]{\overset{\sim}{#1}}
%\newcommand{\legendre}[2]{\left(\frac{#1}{#2}\right)}
%\newcommand{\Ber}{\mathrm{Ber}} % loi de Bernoulli




\begin{document}

\LARGE
Activité bare-metal \hfill Télécom Robotics\\\\
\small Adressez toute faute, demande de précision ou d'ajout à Pierre Gimalac [\href{mailto:pierre.gimalac@gmail.com}{pierre.gimalac@gmail.com}]\\
ou directement sur le dépôt github [\href{https://github.com/TelecomParistoc/bare-metal-activity}{https://github.com/TelecomParistoc/bare-metal-activity}]

\normalsize
\renewcommand{\contentsname}{Sommaire}
\thispagestyle{empty}
\tableofcontents
\thispagestyle{empty}

\section{Introduction}
L'objectif de cette activité est d'apprendre la programmation \textit{bare-metal}, c'est à dire sur microcontroleur, sans système d'exploitation pour nous faciliter la tâche.\\

En particulier il n'y a presque aucune abstraction avec le matériel à part celles fournies par le processeur, c'est à nous de charger le programme en mémoire et de gérer les interruptions.\\

De plus, il n'y a aucune fonction fournie et pas de gestion des processus donc tout doit se faire dans le même processus (dans un premier temps).\\

Une grosse partie de la programmation sur microcontroleur est de lire la documentation pour connaître le fonctionnement précis de celui-ci. Ce TP vous indiquera quels documents et quelles pages doivent être regardées pour vous éviter de lire des centaines de pages comme le pauvre auteur.\\

De plus, une architecture initiale est fournie pour éviter de passer trop de temps sans même arriver à exécuter un programme (voir \autoref{dependances} \nameref{dependances} et \autoref{fichiers_fournis} \nameref{fichiers_fournis}).\\

L'objectif du TP est d'arriver à allumer une led située sur le microntroleur en appuyant sur l'un des boutons.

\newpage

\section{Prérequis}
\subsection{\label{dependances}Dépendances}
\subsection{Langage C}
\section{Quelques points théoriques}
\section{\label{fichiers_fournis}Les fichiers fournis}
\section{Le vif du sujet}


\end{document}
